\documentclass[conference]{IEEEtran}
%\usepackage{ifpdf}
% *** CITATION PACKAGES ***
%
%\usepackage{cite}
% *** GRAPHICS RELATED PACKAGES ***
%
\ifCLASSINFOpdf
  % \usepackage[pdftex]{graphicx}
  % \graphicspath{{../pdf/}{../jpeg/}}
  % \DeclareGraphicsExtensions{.pdf,.jpeg,.png}
\else
  % \usepackage[dvips]{graphicx}
  % \graphicspath{{../eps/}}
  % \DeclareGraphicsExtensions{.eps}
\fi
% *** MATH PACKAGES ***
%
%\usepackage[cmex10]{amsmath}
%\usepackage{algorithmic}
%\usepackage{array}
%\usepackage{mdwmath}
%\usepackage{mdwtab}
%\usepackage{eqparbox}
%\usepackage[tight,footnotesize]{subfigure}
%\usepackage[caption=false]{caption}
%\usepackage[font=footnotesize]{subfig}
%\usepackage{fixltx2e}
%\usepackage{stfloats}
%\usepackage{url}
\hyphenation{op-tical net-works semi-conduc-tor}


\begin{document}
%
% paper title
% can use linebreaks \\ within to get better formatting as desired
\title{Bare Demo of IEEEtran.cls for Conferences}


% author names and affiliations
% use a multiple column layout for up to three different
% affiliations
\author{\IEEEauthorblockN{Raphael Sofaer}
\IEEEauthorblockA{Email: rjs454@nyu.edu}
}

% make the title area
\maketitle


\begin{abstract}
%\boldmath
We examine changes in the environment of wireless communications
research over the last two decades by comparing a paper written in 1991 \cite{oldpaper:rapp}
with a recent 2011 paper \cite{bigpaper:rapp} and reflect on some interesting current research topics.
\end{abstract}
\IEEEpeerreviewmaketitle

\section{Introduction}
In 1991, according to ``The Wireless Revolution'', there were 6.3 million
cellular subscribers in the USA.  In 2011, when the more recent
``State of the art in 60 GHz\ldots'' was written, almost every adult in the
United States had a mobile phone.  The earlier article predicts an explosion
of personal wireless devices, and briefly presents a tool for modeling spread and 
attenuation of the wireless signals that would be used for those devices,
called SIRCIM.  The later article is concerned with a new frequency band
in a world where hardly anyone is without a wireless device at any time.
It examines each component of a wireless radio system, and surveys the
state of the art in that component in the 60GHz band.\\

\subsection{Local Bands}
``The Wireless Revolution'' is concerned with arguing that wireless networks
would soon rapidly expand in the 900MHz and other cellular bands, and
presenting the work of the MPRG research group at the Virginia Polytechnic
Institute towards making that expansion practical, in the form of modeling
software.  The new piece of spectrum enables more bandwidth, more users, and 
new applications.  In 2001, the 60GHz band is filling much the same role.  
The prospect of vast amounts of bandwith inspires new applications, like 
wireless connections that only reach through one room, in order to connect
to highly local devices.  

\subsection{Technical Detail}
The 1991 paper is mostly historical, while the 2011 paper is mostly about
specific engineering efforts.

\subsubsection{Software vs Hardware}
The technical part of the 1991 paper is mostly about software,
while the 2011 paper is very concerned with specific hardware and
hardware production techniques.

\subsection{Magazine article vs Journal article}
The 1991 paper is a magazine article, rather than a technical paper.
It doesn't have an abstract and is much more casual reading.  The 2011 paper
is much more technical, with the abstract and keyword listing needed for a
journal article.

\section{Possible Research Interests}
\subsection{Environmental Monitoring}
As the cost of electronics goes down, applications that might have been
absurdly costly in the past seem plausible.  One such application might be
widespread monitoring of our living environment in radically more detail
than we have had in the past.  Today, a new arrival in New York City 
looking for an apartment might look at crime heatmaps, maps of businesses,
maps of language use, and typical price maps.  What if that renter could 
look at maps of how noise levels on his prospective block change over a typical
day?  What if he could know how air pollution in his area changes when
the wind comes over the Gowanus Canal?\\
In Java and Indonesia, researchers are considering using low-cost battery powered
sensing kits for vital warning of natural disasters as well as ecological
sensing \cite{envmon:wira}.  Their prototype kits monitor water level and provide
a camera feed, with the goal of being an early warning system for floods coming
down a river.  Other applications include water quality monitoring, seismic 
monitoring, and animal monitoring.  The researchers use a package connected by wifi to their 
university network, but they plan on using longer range radio to connect
more distant sensors to the internet.\\
In another project, researchers from New South Wales and Portland
State University are working to use the already widely distributed smartphones
in urban areas to monitor street noise \cite{envmon:rana}.  Each reading comes with GPS or other
location data as well as noise level data, so even though there are no permenant
sensors, it is possible to have consistent monitoring of any area.  Records are
gathered constantly, than transmitted in batches to a central server. \\
Today, new funding options like Kickstarter allow a project to be useful on a 
small scale while building a foundation to mass-produce a small device.  In 
one example, an IndieGogo project called uBiome \cite{envmon:rich} aims to 
create a sampling kit that lets anyone pay a small amount to get a sampling kit
and characterize their stomach, skin and gut bacteria.  This is
useful today, but if more data is gathered, it becomes more useful as what is
normal becomes clearer.  Similarly, if there was a cheap sensor that anyone could buy, put a battery into,
duct tape to a wall, and forget about, an environmental monitoring system might
be able to smoothly scale from slightly useful to comprehensive. 
A package with a noise level, particulate, and GPS sensors could use a cellular
connection if necessary, or wifi, or for a long batter life, APRS telemetry to 
transmit data to a central server.  Anyone interested in monitoring their environment
could purchase either sensor nodes that would connect directly to the internet or
to some internet-connected middle device and add their data to a central repository
for analysis.  If costs were low enough, anyone interested could look at the
difference in noise level and air quality not only between different parts of
a city, but between different floors above the street.  A sensor powered by 
a few batteries could last for at least a year, and by transmitting voltage data, the
sensor could effectively ask the central repository to call a nearby volunteer
to come by with a few new AAs.  This sort of publicly available low maintenance
sensor network could keep residents of any area informed about their environment
without needing expensive studies or equipment to understand their living space.
\subsection{Mesh Networking}
Aside from buggy and mostly useless support for 'ad-hoc' wireless networking,
today's consumer wireless data capable almost all connect to central hubs and 
awkwardly switch as the user moves out of range of the old network. \\

Naval Postgraduate School research: OPTIMAL TRANSMITTER PLACEMENT IN WIRELESS MESH NETWORKS\\
MobiMESH project in Milan.\\
OLPC Mesh Networking

\begin{thebibliography}{1}

\bibitem{oldpaper:rapp}
Theodore S. Rappaport, \emph{The Wireless Revolution}, IEEE Communications Magazine, November 1991.
\bibitem{bigpaper:rapp}
Theodore S. Rappaport, James N. Murdock, and Felix Gutierrez, \emph{State of the Art in 60-GHz Integrated Circuits and Systems for Wireless Communications}, Proceedings of the IEEE, Vol. 99, No. 8, August 2011.
\bibitem{envmon:wira}
Wirawan et al., \emph{Design of Low Cost Wireless Sensor Networks-Based Environmental Monitoring System for Developing Country}, Proceedings of APCC, 2008.
\bibitem{envmon:rana}
Rana et al.,\emph{Ear-Phone: An End-to-End Participatory Urban Noise Mapping System}, Proceedings of the 9th ACM/IEEE International Conference on Information Processing in Sensor Networks, 2010.
\bibitem{envmon:rich}
Jessica Richman, \emph{uBiome – Mapping Your Microbiome}, http://ubiome.com/
\end{thebibliography}




% that's all folks
\end{document}


